\documentclass[11pt,a4paper,twoside,openright]{report}

\usepackage[top=25mm,bottom=25mm,right=25mm,left=30mm,head=12.5mm,foot=12.5mm]{geometry}
\let\openright=\cleardoublepage

\input{macros}

\def\NazevPrace{SimonPose}
\def\Trida{4.B}
\def\AutorPrace{Šimon Brávek}
\def\DatumOdevzdani{2026}

% Vedoucí práce: Jméno a příjmení s~tituly
\def\Vedouci{Jiří Matas}

% Studijní program a obor
\def\StudijniProgram{studijní program}
\def\StudijniObor{studijní obor}

% Text čestného prohlášení
\def\Prohlaseni{Prohlašuji, že jsem svou práci vypracoval samostatně a použil jsem pouze prameny a literaturu
uvedené v~seznamu bibliografických záznamů. Nemám žádné námitky proti zpřístupňování této práce v~souladu se
zákonem č. 121/2000 Sb. o~právu autorském, o~právech souvisejících s~právem autorským a
o~změně některých zákonů (autorský zákon) ve znění pozdějších předpisů.}

% Text poděkování
\def\Podekovani{%
Děkuji Jiřímu Matasovi za skvělou příležitost strávit měsíc v létě na katedře strojového učení a být součástí soudobého výzkumu a za jeho skvělé nápady v tomto oboru. Bezpochyby tato zkušenost formovala mé rozhodnutí studovat obor AI ma MFF UK a propůjčila mi vášeň k nejnovějším technologiím. Děkuji také Matěji Suchánkovi jako člověku, co mě provedl tou nejtěžší asimilací s prácí s grafickými kartami a modely jež jsem používal. Děkuji také lidem z katedry, jež ke mě byli přátelští a v neposlední řadě Igoru Vujovičovi za rozvíjení informatiky na našem gymnáziu i přez mnohé překážku. 
}

% Abstrakt česky
\def\Abstrakt{%

}

% Abstrakt anglicky
\def\AbstraktEN{%
}

% 3 až 5 klíčových slov
\def\KlicovaSlova{počítačové vidění, odhad lidské pozice, DensePose, SMPL/SMPL-X, \\rekonstrukce z 2D do 3D, hluboké učení, vylepšení modelu}
% 3 až 5 klíčových slov anglicky
\def\KlicovaSlovaEN{computer-vision, human-pose-estimation, DensePose, SMPL/SMPL-X, \\2D-to-3D reconstruction, deep-learning, model improvement}


\begin{document}

\include{titlepage}

% Obsah
\setcounter{tocdepth}{2}
\tableofcontents

\chapter{Teoretická část}
\pagestyle{fancy}

V mojí práci se zaměřuji na vylepšení odhadu lidské pozice na základě jednoho vstupního obrázku. Jedná se o disciplínu počítačového vidění, která se vyvinula relativně nedávno - nejvíc při zve

\section{První sekce teoretické části}

\lipsum


\chapter{Implementace}

Druhá kapitola obsahuje detailní informace o tom, jak probíhala implementace. Zde se objeví zdůvodnění výběru technologií, řešení problémů, na které jste narazili, informace o použitých knihovnách apod. Pochvalte se, nikdo to za Vás neudělá. Přiznejte chyby, není to ostuda.

\section{Ukázka sekce}

\lipsum

\chapter{Technická dokumentace}

Poslední kapitola obsahuje informace o tom, jak projekt, který v rámci maturitní práce vznikl, nainstalovat, spustit a používat.

\section{Ukázka sekce}

\lipsum[5]

\subsection{A jedné podsekce}

\lipsum

\section{A další sekce}

\lipsum

\chapter*{Závěr}
\pagestyle{empty}
\addcontentsline{toc}{chapter}{Závěr}

Závěr obsahuje shrnutí práce a vyjadřuje se k míře splnění jejího zadání. Dále by se zde mělo objevit sebehodnocení studenta a informace o tom, co nového se naučil a jak vnímal svou práci na projektu.

%%% Seznam použité literatury
\nocite{einstein}\nocite{latexcompanion}\nocite{knuthwebsite}
\printbibliography[title={Seznam použité literatury},heading={bibintoc}]

%%% Seznam obrázků
\openright
\listoffigures
\addcontentsline{toc}{chapter}{Seznam obrázků}

%%% Seznam tabulek
\clearpage
\listoftables
\addcontentsline{toc}{chapter}{Seznam tabulek}

%%% Přílohy k práci, existují-li. Každá příloha musí být alespoň jednou
%%% odkazována z vlastního textu práce. Přílohy se číslují.

%\part*{Přílohy}
%\appendix

\end{document}
