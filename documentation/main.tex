\documentclass[11pt,a4paper,twoside,openright]{report}

\usepackage[top=25mm,bottom=25mm,right=25mm,left=30mm,head=12.5mm,foot=12.5mm]{geometry}
\let\openright=\cleardoublepage

\input{macros}

\def\NazevPrace{SimonPose}
\def\Trida{4.B}
\def\AutorPrace{Šimon Brávek}
\def\DatumOdevzdani{2026}

% Vedoucí práce: Jméno a příjmení s~tituly
\def\Vedouci{Jiří Matas}

% Studijní program a obor
\def\StudijniProgram{studijní program}
\def\StudijniObor{studijní obor}

% Text čestného prohlášení
\def\Prohlaseni{Prohlašuji, že jsem svou práci vypracoval samostatně a použil jsem pouze prameny a literaturu
uvedené v~seznamu bibliografických záznamů. Nemám žádné námitky proti zpřístupňování této práce v~souladu se
zákonem č. 121/2000 Sb. o~právu autorském, o~právech souvisejících s~právem autorským a
o~změně některých zákonů (autorský zákon) ve znění pozdějších předpisů.}

% Text poděkování
\def\Podekovani{%
Děkuji Jiřímu Matasovi za skvělou příležitost strávit měsíc v létě na katedře strojového učení a být součástí soudobého výzkumu a za jeho skvělé nápady v tomto oboru. Bezpochyby tato zkušenost formovala mé rozhodnutí studovat obor AI ma MFF UK a propůjčila mi vášeň k nejnovějším technologiím. Děkuji také Matěji Suchánkovi jako člověku, co mě provedl tou nejtěžší asimilací s prácí s grafickými kartami a modely jež jsem používal. Děkuji také lidem z katedry, jež ke mě byli přátelští a v neposlední řadě Igoru Vujovičovi za rozvíjení informatiky na našem gymnáziu i přez mnohé překážku. 
}

% Abstrakt česky
\def\Abstrakt{%

}

% Abstrakt anglicky
\def\AbstraktEN{%
}

% 3 až 5 klíčových slov
\def\KlicovaSlova{počítačové vidění, odhad lidské pozice, DensePose, SMPL/SMPL-X, \\rekonstrukce z 2D do 3D, hluboké učení, vylepšení modelu}
% 3 až 5 klíčových slov anglicky
\def\KlicovaSlovaEN{computer-vision, human-pose-estimation, DensePose, SMPL/SMPL-X, \\2D-to-3D reconstruction, deep-learning, model improvement}


\begin{document}

\include{titlepage}

% Obsah
\setcounter{tocdepth}{2}
\tableofcontents

\chapter*{Úvod}
\addcontentsline{toc}{chapter}{Úvod}

\section{Analýza postavy v digitálním světě}

V současné informatice a počítačovém vidění (Computer Vision) představuje automatické porozumění lidského pohybu jednu z největších výzev. Nejde již o prostou detekci zda se v obrázku nachází člověk, či kolik jich je, ale o snahu přenést lidskou biomechaniku do digitálního prostoru. Schopnost přesně interpretovat lidskou pózu z běžného 2D obrazu (např. z mobilního telefonu) otevírá dveře aplikacím, které byly dříve nemyslitelné bez drahých studiových systémů pro snímání pohybu (Motion Capture).

Tyto technologie se dnes nachází již v mnoha podobách a můžou tak pomoci v celé řadě aplikací. Dnes již můžeme najít rozpoznání člověka a obličeje v mobilní aplikaci galerie, kde napomáhá chytrému třízení fotek nebo v bezpečnostních systémech pro detekci vetřelce. Detekce konkrétní pozice, tedy kde se nachází jednotlivé klouby, může zase pomoci zastavit automatický vozík před kolizí s člověkem nebo zavolat pomoc pokud uvidí nehybně ležícího chodce. 

Kromě přesnosti těchto metod se vědci čím dál více zaměřují také na rychlost. Mnohé aplikace totiž potřebují živou analýzu videa a to často i z několika kamer najednou. Tak tomu je u autonomního řízení automobilů, které se ukazuje být velkým tahounem tohoto odvětví.

Právě rozmanitost této disciplíny, její užitečnost a také možnost setkat se s ní do hloubky na FEL ČVUT mě vedlo k její volbě jako tématu mé maturitní práce.

\section{Problematika}

Současné SOTA (State-of-the-Art) modely sice dosahují vynikajících výsledků na laboratorních datasetech, ale často selhávají v reálných podmínkách (in-the-wild). Mezi kritické faktory patří:

\begin{itemize}
    \item \textbf{Zákryty (Occlusions):} Části těla zakryté předměty nebo jinými částmi těla.
    \item \textbf{Perspektivní zkreslení:} Extrémní úhly kamery, které deformují vizuální proporce těla.
    \item \textbf{Vizuální šum:} Volné oblečení nebo špatné světelné podmínky.
    \item \textbf{Propletení více těl:} Situace v davu, kde není jasné, komu patří jaká část těla a jak je správně rozdělit. 
\end{itemize}

Právě kombinace modelu SMPL s technologií DensePose (která mapuje pixely přímo na povrch těla) nabízí unikátní cestu, jak tyto problémy řešit skrze hustou korespondenci dat.

\section{Cíl práce}

Cílem této maturitní práce není vytvořit nový, revoluční model, který by překonal stávající vědecké rekordy. Ambicí je metodický průzkum možností, jak stávající proces fittingu (pasování) modelu SMPL do dat z DensePose zpřesnit a učinit jej odolnějším (robustnějším).

Zvolená metodika se opírá o princip Fail-Fast:

\begin{itemize}
    \item Rychlá formulace hypotéz o vylepšení optimalizačního procesu.
    \item Implementace prototypů a jejich testování na hraničních případech.
    \item Analýza selhání jakožto hlavního zdroje poznání.
\end{itemize}

Výsledkem práce je ucelený přehled vyzkoušených metod, jejich kritické zhodnocení a dokumentace slepých i perspektivních uliček, které mohou sloužit jako inspirace pro další vývoj v oblasti monokulární 3D rekonstrukce člověka.

\chapter{Teoretická část}
\pagestyle{fancy}

\section{Základní teorie} 

\subsection{Co je to model?}

V této práci se často budu věnovat programům, které využívají metodu strojového učení k detekci a odhadu pozice člověka v obrázku, který nazýváme model. Tento program má mnoho parametrů na základě nichž přetváří vstup na výstup. Nejdříve projde procesem, které nazýváme trénování. Během toho se jako vstup použije obrázek a program k němu vygeneruje souřadnice, kde odhaduje klouby, na základě prvního nastavení parametrů, neboli inicializace. Inicializace bývá velmi složitá a často je předmětem celých studií \cite{mishkin2016goodinit} a nebudeme se jí věnovat v této práci. .Tento výsledek se pak porovná s ukázkovým příkladem pro tento obrázek, tedy pro anotaci, která byla předtím připravena člověkem. 

Dále zvolíme metodu pro výpočet chyby výstupu (loss). Proces generování výstupu opakujeme mnohokrát pro velké množství obrázků a anotací toho. Skupině těchto dat říkáme trénovací dataset. Je nutné mít velké množtví obrázků a anotací ještě před začátkem trénování. Ty pochází v drtivé většině případů od lidských anotátorů a jsou tak velmi drahé. Navíc jsou předmětem lidské chybovosti. Součtem, nebo jinou metodou tak spočítáme ztrátovou funkci (loss funcion) jako metriku chybovosti odhadu našeho modelu od anotací

\subsection{Druhy detekcí}

K tomuto účelu se využívá hned několik metod. Tou úplně nejzákladnější jsou rámečky k ohraničení detekovaných lidí. skládají se ze dvou souřadnic.


\subsection{Odhad 2D a jeho limitace}

Tradiční metody odhadu pózy se po léta soustředily na tzv. sparse keypoints -- detekci klíčových bodů, jako jsou lokty, kolena či ramena -- které detekují jako dvojici souřadnic v daném obrázku. Přestože jsou tyto modely (např. OpenPose) rychlé a efektivní, trpí zásadním nedostatkem: ztrátou prostorové informace a tělesného objemu. Zatímco mají perfektní výsledky v laboratorních podmínkách, selhávají v reálných situacích (in-the-wild), kde jim 2D obrázek neposkytuje dost jasné informace pro detekci pozice. Mezi faktory ovlivňující přesnost výstupu patří:

\begin{itemize}
    \item \textbf{Zákryty (Occlusions):} Části těla zakryté předměty nebo jinými částmi těla.
    \item \textbf{Perspektivní zkreslení:} Extrémní úhly kamery, které deformují vizuální proporce těla.
    \item \textbf{Vizuální šum:} Volné oblečení nebo špatné světelné podmínky.
    \item \textbf{Propletení více těl:} Situace v davu, kde není jasné, komu patří jaká část těla a jak je správně rozdělit. 
\end{itemize}

\subsection{Od 2D bodů k 3D objemu}

Řešení těchto složitých případu významně napomáhá představa 3D modelu těla a jeho pozice na dané scéně. Vezmeme-li situaci člověka skákajícího na lyžích zespudu, jeho tělesné proporce budou zcela nestandardní (malá hlava, velké nohy, spousta zakrytých částí těla). Pokud si ale představím model lidského těla a promítnu ho do obrázku, pak jsem značně omezen a najít správné orientace končetin se značně zjednoduší. Představa těla ve 2D mi naopak dovoluje zvažovat pozice, jež by byli lidské tělo zcela nepřirozené, či anatomicky nemožné. Modely, které spolu s pozicí kloubů odhadují také 3D orientaci těla se díky tomu stávají lepší v samotném odhadování pozice kloubů.

Samotné rozpoznávání 2D pozice se dostává ke svým limitům také proto, že jejich přesnost se blíží datům na kterých jsou trénovány. 
\cite{moravkova2025}. 
Další k rozpoznání 3D pozice těla je zle



\section{Skinned Multi-Person Linear Model (SMPL)}

Přechod od 2D chápání člověka k pochopení objemové struktury lidského těla si vyžaduje zcela nové nástroje. Jedním z nich je způsob jak modelovat lidské tělo v prostoru. Abychom mohli tělo vyrendrovat na obrazovku pomocí standardních postupů, musíme povrch těla zapsat jako množinu bodů a zapamatovat si všechny možné trojice bodů tak, aby nám vzikla síťovina (mesh) reprezentující povrch 3D tělesa. 

Namodelovat realistické lidcké tělo na základě bodů je velmi náročná disciplína. My bychom takové tělo chtěli modelovat automaticky, v reálném čase a s tělesnými proporcemi a pozicí, jakou si zadáme. Proto vznikl SMPL model, který na základě parametrů $\beta$ a $\theta$ vytvoří síťovinu libovolného člověka. 

Celý model těla vzniká takto: 


\begin{itemize}
    \item \textbf{T-pose:} Souřadnic bodů kanonického těla v klidu (tedy univerzálního zvolenéhe těla, které zobrazuje průměrného člověka ve všech proporcích). 
    \item \textbf{Joints:} Polohy všech kloubů požadované pozice těla. Každý bod na těle je posunut na tomto základě. Poloha všech kloubů je plně závislá na parametru $\theta$. 
    \item \textbf{Omega:} Reprezentuje tělesné proporce konkrétního člověka.
    \item \textbf{Propletení více těl:} Situace v davu, kde není jasné, komu patří jaká část těla a jak je správně rozdělit. 
\end{itemize}

$M(\beta, \theta) = W(T_p(\beta, \theta), J(\beta), \theta, \mathcal{W})$
Generativní proces modelu SMPL (od šablony přes tvarové a pózové deformace až po skinning) je schematicky znázorněn na obr.~\ref{fig:smpl-stage-1}--\ref{fig:smpl-stage-3}.


Začne se se souřadnicemi bodů kanonického těla $\bar{T}$ v klidu (tedy univerzálního zvolenéhe těla, které zobrazuje průměrného člověka ve všech proporcích). Ke každému bodu se následně přičte posunutí na základě parametru tělesných proporcí $\beta$ příspěvkem $B_s(\beta)$. Tak vznikne T-pose správných tělesných proporcí. Také vypočteme počáteční pozici kloubů $J(\beta)$ na základě tělesných proporcí. 

\begin{figure}[H]
    \centering
    \includegraphics[width=0.95\textwidth]{img/stage_1.png}
    \caption{SMPL -- stage 1}
    \label{fig:smpl-stage-1}
\end{figure}

Při pohybu se naše tkáně napínají a ohýbají a s tím musíme počítat i u počítačového modelu člověka. Dále tedy upravíme proporce těla tak, aby odpovídali po pozici do které chceme model dostat tím, že k modelu přidáme příspěvek $B_P(\theta)$ a vznikne tak pozice $T_p(\beta, \theta)$.

\begin{figure}[H]
    \centering
    \includegraphics[width=0.95\textwidth]{img/stage_2.png}
    \caption{SMPL -- stage 2}
    \label{fig:smpl-stage-2}
\end{figure}

Nakonec se vše poskládá dohromady pomocí funkce $W(\cdot)$ ještě s konečnou pozicí končetin $\theta$ a maticí $\mathcal{W}$. Tato matice spolu s $B_s(\beta)$ a $B_p(\theta)$ jsou natrénované hodnoty na tisících lidských skenů. 

\begin{figure}[H]
    \centering
    \includegraphics[width=0.95\textwidth]{img/stage_3.png}
    \caption{SMPL -- stage 3: aplikace linear blend skinningu $W(\cdot)$ a výsledná póza těla v prostoru.}
    \label{fig:smpl-stage-3}
\end{figure}

Tento model má výhodu v tom, že parametry $\theta$ a $\beta$ usměrní výslednek tak, aby bylo velmi těžké zdeformovat obrázek do nelidských proporcí. Zároveň však dokáže popsat celou řadu lidských těl. Je díky tomu vhodná pro

Tato knihovna je pro komerční účely zpoplatněná, ale pro vědecké účely je zdarma a stačí se registrovat na https://smpl.is.tue.mpg.de/index.html. Díky tomu se z ní stal standard v oblasti počítačového vidění. 


\section{DensePose}



\section{Navržené metody}

\subsection{Vzdálenost bodu od těla}



\subsection{Euklidovká vzdálenost}



\subsection{Zaměření na přesnost}



\chapter{Implementace}

Druhá kapitola obsahuje detailní informace o tom, jak probíhala implementace. Zde se objeví zdůvodnění výběru technologií, řešení problémů, na které jste narazili, informace o použitých knihovnách apod. Pochvalte se, nikdo to za Vás neudělá. Přiznejte chyby, není to ostuda.

\section{Ukázka sekce}

\lipsum

\chapter{Technická dokumentace}

Poslední kapitola obsahuje informace o tom, jak projekt, který v rámci maturitní práce vznikl, nainstalovat, spustit a používat.

\section{Ukázka sekce}

\lipsum[5]

\subsection{A jedné podsekce}

\lipsum

\section{A další sekce}

\lipsum

\chapter*{Závěr}
\pagestyle{empty}
\addcontentsline{toc}{chapter}{Závěr}

Závěr obsahuje shrnutí práce a vyjadřuje se k míře splnění jejího zadání. Dále by se zde mělo objevit sebehodnocení studenta a informace o tom, co nového se naučil a jak vnímal svou práci na projektu.

%%% Seznam použité literatury
\printbibliography[title={Seznam použité literatury},heading={bibintoc}]

%%% Seznam obrázků
\openright
\listoffigures
\addcontentsline{toc}{chapter}{Seznam obrázků}

%%% Seznam tabulek
\clearpage
\listoftables
\addcontentsline{toc}{chapter}{Seznam tabulek}

%%% Přílohy k práci, existují-li. Každá příloha musí být alespoň jednou
%%% odkazována z vlastního textu práce. Přílohy se číslují.

%\part*{Přílohy}
%\appendix

\end{document}
